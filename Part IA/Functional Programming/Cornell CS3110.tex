\documentclass[12pt,a4paper]{article} % uncomment, if revtex is not installed
\usepackage[utf8]{inputenc}
\usepackage{amsmath}
\usepackage{amsfonts}
\usepackage{amssymb}
\usepackage{enumitem}
\usepackage{color}
\usepackage[english]{babel}
\usepackage{graphicx}
\usepackage{bm}
\usepackage{wasysym}
\usepackage{natbib}
\usepackage{fixmath}
\usepackage{comment}
\usepackage{fancyhdr}
\usepackage{pgfplots}
\usepackage{listings}
\pagestyle{fancyplain}
\usepackage{graphicx}
\graphicspath{ {./} }
\fancyhf{}
\setlength{\parindent}{0in}
\lstset{
tabsize = 4, %% set tab space width
showstringspaces = false, %% prevent space marking in strings, string is defined as the text that is generally printed directly to the console
numbers = left, %% display line numbers on the left
commentstyle = \color[rgb]{0,0.5,0}, %% set comment color
keywordstyle = \color{blue}, %% set keyword color
stringstyle = \color{red}, %% set string color
rulecolor = \color{black}, %% set frame color to avoid being affected by text color
basicstyle = \small \ttfamily , %% set listing font and size
breaklines = true, %% enable line breaking
numberstyle = \tiny,
}
\lhead{ \fancyplain{}{Ivin Lee, il315@cam.ac.uk} }
\rhead{ \fancyplain{}{NST Math 2016}}
\cfoot{ \fancyplain{}{\thepage{}} }

\newcommand{\snt}[1]{\textcolor{magenta}{#1}} % edited by Sergei 
\newcommand{\crsid}[1]{\textcolor{red}{#1}} % edited by crsid
\newcommand{\sntdel}[1]{\textcolor{green}{\sout{#1}}} % deleted by Sergei

\newcommand{\problem}[1]{\subsection*{#1}
\setcounter{equation}{0}}
\newcommand{\question}[1]{\problem{Q. #1}}
\newcommand{\pproblem}[1]{\subsubsection*{(#1)}}
\author{Ivin Lee}
\title{Cornell CS 3110 - Functional Programming}

\begin{document}
%%%%%%%%%%%%%%%%%%%%%%%%%%%%%%%%%%%%%
\maketitle
No, I'm not from Cornell. But I think this set of lectures on OCaml is really useful for learning functional programming, which is why I am including it in these notes. I studied this course before my Java course because I think it will add value to learn Java course with prior knowledge of the functional programming paradigm.
\section{Introduction}
\textbf{Functional languages: }
\begin{itemize}
	\item define computations as mathematical functions
	\item avoid mutable states
\end{itemize}
The former means that every computation takes in values and outputs values (more on this later).
\\\\
The latter means that you never write $x = x + 1$, because when you do so you are changing (muting) the value of x.
\\\\
\textbf{Imperative languages:}
\begin{itemize}
	\item mutable states
	\item functions have side effects
\end{itemize}
The latter means that in
\begin{lstlisting}[language = Java]
int addValue(Wallet wallet, int change) {
	wallet.money += change;
	return wallet.money;
}
\end{lstlisting}
apart from returning the new value of the wallet, it also changes the value in the Wallet instance.
\\\\
It is claimed that functional programming allows one to write correct code easier, because:
\begin{itemize}
	\item variables never change values
	\item functions never change other things (and cause trouble!)
\end{itemize}
It is possible to write code following the functional programming paradigm in any programming language, but some languages have been designed to make this easier. e.g.OCaml
\\\\
Benefits of OCaml:
\begin{enumerate}
	\item Immutable programming inbuilt (can't accidentally re-write a variable after being declared)
	\item Functions passed as values
	\item \textbf{Automatic type inference} - quite useful from experience
\end{enumerate}
\section{Before We Begin}
You can compile OCaml programs with \verb|ocamlc| and \verb|ocamlopt|. The difference is that \verb|ocamlc| is the bytecode compiler while \verb|ocamlopt| is the native code compiler. For reference, native code is code which works only on your CPU architecture. In contrast, bytecode is CPU independent, and needs another round of parsing to become native code that runs on your computer's hardware.
\\\\
To compile a program, type 

\verb|ocamlopt -o program.exe module1.ml module2.ml|
\\\\
You may also see 

\verb|ocamlopt -c module1.ml module2.ml|

\verb|ocamlopt -o program.exe module1.cmx module2.cmx|
\\\\
This just means that each module is compiled first, then linked up.
\section{Functions}
\textbf{Value} - expression that does not need any further evaluation
\begin{lstlisting}[language = Caml]
let x = if e1 then e2 else e3
let y = e2 +. e3
\end{lstlisting}
There is no need to declare the type as the type is inferred from e2 and e3 (e.g. if e2 and e3 are ints, then the type of x is inferred to be int).
\\\\
The $+.$ is a binary infix operator that takes in two floats and returns a float. In order for the type of y to be inferred correctly, it is necessary to use the operator, which tells the compiler(?) the return type of this expression.
\\\\
\textbf{Function Definitions}
\begin{lstlisting}[language = Caml]
let rec pow x y = 
	if y = 0 then 1
	else x * pow x (y - 1)
\end{lstlisting}
Note that you don't put brackets around the arguments. Also, all functions only return one thing, which matches the mathematical definition of a function.
\begin{lstlisting}[language = Caml]
let rec f x1 x2 ... xn = e
val f : t1 -> t2 -> ... tn -> te
\end{lstlisting}
The second line is output in the command line. Type \verb|t -> u| is the type of a function that takes an input of type t and returns an output of type u.
\\\\
\textbf{Anonymous Functions}
\begin{lstlisting}[language = Caml]
fun x -> x + 1
let inc = fun x -> x = 1
\end{lstlisting}
Can be used when you don't need the function to have a name, e.g. defining the function in another function's argument like in
\begin{lstlisting}[language = Caml]
let f x y = 
  x + y
f ((fun x -> x + 1) 2) 3 (* returns 6*)
\end{lstlisting}
A function is a value. In the above example, $ x+1 $ is not evaluated until $ 2$ was passed to the anonymous function.
\\\\
Note that the anonymous expression syntax is analogous to lambda expressions in math:
\begin{align*}
	\lambda x.e\\
	\text{fun }x -> e
\end{align*}
\textbf{Function Application Operator}
\begin{lstlisting}[language = Caml]
f e (*equivalent to*) e |> f
5 |> inc |> square
\end{lstlisting}
5 is passed to inc and square
\\\\
\textbf{Functions are Values}

Implication: functions can take functions as arguments and can return functions as results
\section{Lists}
Lists are constructed recursively, so the following constructors are equivalent
\begin{lstlisting}[language=Caml]
let x = [1;2;3]
let x = 1::[2;3]
let x = 1::2::[3]
let x = 1::2::3::[]
let x = h::t (* h: head, t: tail)
\end{lstlisting}
The type of x is a' list, where a' is the type of the elements in the tail of the list (note: the type of h is a' but the type of t is a' list!)
\\\\
\textbf{Pattern Matching}
\begin{lstlisting}[language=Caml]
let f inputList = 
	match inputList with
	|[] -> -1
	|h::t -> h
\end{lstlisting}
Basically pattern matching allows you to look at the first element of the list, and if that doesn't satisfy you, you can recurse and look at the first element of the tail list.
\\\\
\textbf{Linked List}

As expected, the linked list data structure is good for sequential access.
\\\\
\textbf{Function Keyword}
Another way to write this is:
\begin{lstlisting}[language=Caml]
let f = function
	|[] -> -1
	|h::t -> h
\end{lstlisting}
This function takes in one argument, matches the argument to a pattern and returns the corresponding value. The main difference is that using \verb|match ... with| can take multiple inputs but \verb|function| only takes in one input.
\\\\
\textbf{Patterns}
\begin{lstlisting}[language=Caml]
a::[] (*matches lists with 1 element*)
a::b::[] (*matches all lists with 2 elements*)
x (*matches anything that has a value*)
_ (*matches everything*)
\end{lstlisting}
\textbf{Pattern Matching Warnings}

Not all cases considered: inexhaustive pattern-match warning

Duplicated cases: unused match case warning
\section{Let Expressions}
Let definitions have been used until now
\begin{lstlisting}[language=Caml]
let x = 2 in x + x
\end{lstlisting}
This returns the value $ x+x$ so this statement is an expression
\begin{lstlisting}[language=Caml]
let x = 2 in x = 1
(fun x -> x + 1) 2
\end{lstlisting}
These two equations are the same. Basically all these expressions are secretly functions!
\\\\
\textbf{Variant}
\begin{lstlisting}[language=Caml]
type day = Sun | Mon | Tue | Wed | Thu | Fri | Sat
let int_of_day d = 
	match d with
	|Sun -> 1
	|Mon -> 2
	|Tue -> 3
	|Wed -> 4
	|Thu -> 5
	|Fri -> 6
	|Sat -> 7
\end{lstlisting}
Each of the Sun/Mon/Tues are called constructors, which are already a value
\\\\
\textbf{Records}

Need to define a record type for type inference
\begin{lstlisting}[language=Caml]
type contact = {name: string; hp: int}
let nick = {name="Nick"; hp=81234567}
nick.name (*returns "Nick"*)
let get_hp m = 
	match m with
	| {name=_; hp = h} -> h
let get_hp m = 
	match m with 
	|{name; hp} -> hp
let get_hp m = 
	match m with 
	|{hp} -> hp
let get_hp m = m.hp
\end{lstlisting}
All the functions are the same, check if you understand how this pattern matching works!
\\\\\
\textbf{Tuples}
\begin{lstlisting}[language=Caml]
(1,2,10) : int*int*int
(true, "Hello") : bool*string
([1;2;3], (0.5, 'X')) : int list * (float*char)
let f t = 
	match t with
	| (x, y, z) -> z
let f t = 
	let (x, y, z) = t in z
let f t = 
	let (_, _, z) = t in z
let f (_, _, z) = z
f (1, 2, 3) = 3
\end{lstlisting}
\textbf{Extended Syntax for Let}

Previously x was used as a variable, but actually any pattern will work
\begin{lstlisting}[language=Caml]
let x = e1 in e2
let rec f x1 ... xn = e1 in e2
\end{lstlisting}
so replacing all the x's with patterns will work as well
\begin{lstlisting}[language=Caml]
let add t = 
	let (x, y, z) = t
	in x + y + z
let add (x, y, z) = x + y + z
\end{lstlisting}
There are built-in functions for accessing first and second element of a tuple (not sure why you need it...)
\begin{lstlisting}[language=Caml]
let fst (x, _) = x
let snd (_, y) = y
\end{lstlisting}
\textbf{Type synonyms}

Can define types so you don't have to type float list list for matrix etc
\begin{lstlisting}[language=Caml]
type point = float * float
type vector = float list
type matrix = float list list
\end{lstlisting}
Of course, you will have to specify the type because the compiler isn't smart enough to read your mind!
\begin{lstlisting}[language=Caml]
type point = float * float
let f x : point =
  let y = x +. x in
  (y, y)
 (* if : point is not written, the type of f will be float -> float*float instead of float -> point!*)
\end{lstlisting}
\textbf{Back to variants}

Recall: variants are enumerated sets of values. In particular, you can have a variant that enumerates values that of different types.
\begin{lstlisting}[language=Caml]
type point = float * float
type shape =
	| Point of point
	| Circle of point * float
	| Rect of point * point

let pt1 : point = (1., 2.)
(*pt1 = point(1., 2.) is invalid!! *)
let pt2 : shape = Point(1., 2.)
let pt3 = Circle (pt1, 2.)
let pt4 = Circle ((1., 2.), 3.)
(*note that since Circle's type is point * float, you don't need to define the tuple as a point since it is already defined!*)
\end{lstlisting}
The data type of shape is called an \textbf{algebraic data type} because it is a \textbf{tagged union} (Google nonintersecting union if you're not sure)
\\\\
This allows you to classify things under other things. This is similar to inheritance with classes in Java. We may combine this with functions to allow a function to take in arguments of various types (patterns), which is similar to overloading in Java.
\begin{lstlisting}[language=Caml]
let area = function
	| Point _ -> 0.0
	| Circle (_,r) -> pi *. (r ** 2.0)
	| Rect ((x1,y1),(x2,y2)) ->
	let w = x2 -. x1 in
	let h = y2 -. y1 in
	w *. h
\end{lstlisting}
\textbf{Recursive Variants}

We may implement the linked list data type with a recursive variant definition.
\begin{lstlisting}[language=Caml]
type intlist = Nil | Cons of int * intlist

let emp = Nil
let l3 = Cons (3, Nil) (* 3::[] or [3]*)
let l123 = Cons(1, Cons(2, l3)) (* [1;2;3] *)

let rec sum (l:intlist) =
	match l with
	| Nil -> 0
	| Cons(h,t) -> h + sum t
	
let rec length = function
	| Nil -> 0
	| Cons (_,t) -> 1 + length t
	(* length : intlist -> int *)

let empty = function
	| Nil -> true
	| Cons _ -> false
	(* empty: intlist -> bool *)
\end{lstlisting}
We may also use \verb|'a| to allow our list to accept other types.
\begin{lstlisting}[language=Caml]
type 'a mylist = Nil | Cons of 'a * 'a mylist
int mylist
\end{lstlisting}
Mylist is known as a \textbf{type constructor}, because it takes a type as an input and returns a type.
\section{Higher-order Programming}
Recall that functions are values, so we can pass them as arguments. This is an example of functions being 'first-class citizens' lol.
\begin{lstlisting}[language=Caml]
let square x = x * x
let quad x = (square x) * (square x)
let twice f x = f (f x)
let quad2 x = twice square x
\end{lstlisting}
The slides didn't mention this, but I believe that the lack of distinction between functions and values is why braces around function arguments were not implemented.
\\\\
\textbf{Map and Fold}

Note: map is not a method to store key - value pairs!
\begin{lstlisting}[language=Caml]
let rec map f = function
	|[] -> []
	| h::t -> (f h) :: (map f t)

let add1 = List.map (fun x -> x + 1)
let list1 = [1; 2; 3; 4]
add1 list1 (*returns [2; 3; 4; 5]*)
\end{lstlisting}
\textbf{Filter}
\begin{lstlisting}[language=Caml]
let rec filter f = function
	|[] -> []
	| h::t -> 
		if f h
		then h::(filter f t)
		else filter f t

let filter1 = List.filter (fun x -> x > 2)
let list1 = [1; 2; 3; 4]
filter1 list1 (*returns [3; 4]*)
\end{lstlisting}
These are called iterators.
\\\\
\textbf{Combining Elements}
\begin{lstlisting}[language=Caml]
let rec combine init op = function
	| [] -> init
	| h::t -> op h (combine init op t)
	
let sum = combine 0 (+)
let concat = combine "" (^)
\end{lstlisting}
Notice once again how all these functions can be ordered around as values. With braces, the code would be more confusing than without, which is quite interesting! Notice also the use of \verb|()| to convert an infix operator to a prefix operator.
\\\\
OCaml has a \verb|fold_right / fold_left| operator to do this explicitly.
\begin{lstlisting}[language=Caml]
let rec fold_right f t acc =
	match t with
	| [] -> acc
	| h2::t2 -> f h2 (fold_right f t2 acc)

List.fold_right f [a;b;c] init
(*computes f a (f b (f c init))*)

let rec fold_left f acc t =
	match t with
	| [] -> acc
	| h2::t2 -> fold_left f (f acc h2) t2
(*note the difference between fold_left and fold_right's implementation - fold_right has to reach the rightmost element before evaluating f but fold_left evaluates the left-most element straight away*)
\end{lstlisting}
\section{Modular Programming}
OCaml uses \textbf{functional data structures} which means that data structures never change (again, to avoid having multi-valued things)
\begin{lstlisting}[language=Caml]
module MyStack = struct
  type 'a stack =
    | Empty
    | Entry of 'a * 'a stack
  let empty = Empty
  let is_empty s = s = Empty
  let push x s = Entry (x, s)
  let peek = function
    | Empty -> failwith "Empty"
    | Entry(x,_) -> x
  let pop = function
    | Empty -> failwith "Empty"
    | Entry(_,s) -> s
end

let test1 = MyStack.empty;;
let test2 = MyStack.push 1 test1;;
let test3 = MyStack.push 2 test2;;

(*returns*)
val test3 : int MyStack.stack =
  MyStack.Entry (2, MyStack.Entry (1, MyStack.Empty))
\end{lstlisting}
\begin{itemize}
	\item Note the \verb|module...struct| keywords; which is equivalent to classes in Java. Specifically, a module creates a new \textbf{namespace}.
	\item In particular, you need to define the type which is needed for type inference.
	\item Note the last 3 lines: it is not possible to push an element into the original stack, because the stack is not mutable. (If you tried to assign \verb|test1 = MyStack.push 1 test1| you would get \verb|false|! The equate sign is interpreted as checking equality.)
\end{itemize}
\textbf{Function Signature}

It is possible to define the function signature, i.e. what types of variables it can input and output, as well as what functions to be accessed from outside the module.

The 2 benefits are:
\begin{itemize}
	\item input and output types are correct
	\item unnecessary methods are hidden
\end{itemize}
\begin{lstlisting}[language=Caml]
module type S1 = sig
  val x:int
  val y:int
end
module M1 : S1 = struct
  let x = 42
end
(*Error: Signature mismatch:
Modules do not match: sig val x : int end is not included in S1
The value `y' is required but not provided*)
module type S2 = sig
  val x:int
end
module M2 : S2 = struct
  let x = 42
  let y = 7
end
M2.y
(*Error: Unbound value M2.y*)
\end{lstlisting}
\textbf{Abstract Types}

There is one more step to abstraction: not revealing the data structure in the module signature.
\begin{lstlisting}[language=Caml]
module type ListStackSig = sig
  val empty : 'a list
  val is_empty : 'a list -> bool
  val push : 'a -> 'a list -> 'a list
  val peek : 'a list -> 'a
  val pop : 'a list -> 'a list
end

module ListStack = struct
  let empty = []
  let is_empty s = s = []
  let push x s = x :: s
  let peek = function
    | [] -> failwith "Empty"
    | x::_ -> x
  let pop = function
    | [] -> failwith "Empty"
    | _::xs -> xs
end
  (*client's code*)
let x = ListStack.empty;;
let y = ListStack.push 1 x;;
let z = (2 :: y);; 
\end{lstlisting}
In this example, after realising that \verb|ListStack| is implemented with list, the client decides to use the \verb|::| function to concatenate lists. However, if you decide to change the type used in \verb|ListStack| to something else, all of the client's code that uses \verb|::| will break. Hence, the type used in implementation shouldn't be revealed.
\begin{lstlisting}[language=Caml]
module type Stack = sig
  type 'a t
  val empty : 'a t
  val is_empty : 'a t -> bool
  val push : 'a -> 'a t -> 'a t
  val peek : 'a t -> 'a
  val pop : 'a t -> 'a t
end

module ListStack : Stack = struct
  type 'a t = 'a list (*this line informs the compiler that all the ts refer to lists*)
      ...
\end{lstlisting}
Here, we give an arbitrary type \verb|stack| so the client would not know what type we have used. By convention \verb|t| is used to refer to an abstract type.
\section{Functors}
\textbf{Interface Inheritance}
\begin{lstlisting}[language=Caml]
module type Ring = sig
  type t
  val zero : t
  val one : t
  val add : t -> t -> t
  val mult : t -> t -> t
  val neg : t -> t
end
module type Field = sig
  include Ring
  val div : t -> t -> t
end

module FloatRing = struct
  type t = float
  let zero = 0.
  let one = 1.
  let add = (+.)
  let mult = ( *. )
  let neg = (~-.)
end
module FloatField = struct
  include FloatRing
  let div = (/.)
end
\end{lstlisting}
So \verb|include| works for both module signatures and modules.
\\\\
\textbf{Functors}
\\
Functors take structures as inputs and output another structure. So even modules are treated like functions!
\begin{lstlisting}[language=Caml]
module type X = 
sig val x : int 
end

module IncX (M : X) = struct
  let x = M.x + 1
end

(*in anonymous form as well*)
module IncX = functor (M : X) -> struct
  let x = M.x + 1
end

module A = struct let x = 0 end
module B = IncX(A) (* B.x is 1 *)
module C = IncX(B) (* C.x is 2 *)
\end{lstlisting}
One use of functors is to test whether modules of a certain signature are working.
\begin{lstlisting}[language=Caml]
assert (MyStack.(empty |> push 1 |> peek) = 1)
(*The output of MyStack.empty is piped to MyStack.push 1 using the pipe operator. Assert throws an exception if the expression evaluates to false.*)

(*instead of*)
assert (MyStack.(
    empty |> push 1 |> peek) = 1) ;;
assert (ListStack.(
    empty |> push 1 |> peek) = 1);;
assert (WhateverStack.(
    empty |> push 1 |> peek) = 1)
    
(*we could write*)
module StackTester (S:StackSig) = struct
  assert (S.(empty |> push 1 |> peek) = 1)
end
module MyStackTester
  = StackTester(MyStack)
module ListStackTester
  = StackTester(ListStack)
  
(*note that this works because when instantiating the StackTester module, it evaluates each expression within it, even if it isn't a let x = ... expression.*)
\end{lstlisting}
\verb|include| works with functors as well... But to be fair it isn't surprising as a functor is just another module.
\begin{lstlisting}[language=caml]
module type Sig = sig ... end
module Ext (M:Sig) = struct
  include M
  let f =
    ...
end
\end{lstlisting}
\section{Abstraction and Specification}
You should document your code properly if you want it to be readable. The following sections have guidelines on how to write specifications to make it easy for someone else to read.
\begin{lstlisting}[language=caml]
(**
 * returns: [hd lst] is the head of [lst].
 * example: hd [1; 2; 3] is 1.
 * requires: [lst] is non-empty.
 * raises: [Failure "hd"] if [lst] is empty.
 * effects: ...
 *)
val hd : 'a list -> 'a
(*this last line would be in your module signature*)
\end{lstlisting}
The starry formatting allows html or others to parse the comments. The square brackets will format the words as \verb|words|.
\section{Abstraction Functions and Representation Invariants}
Not critical to the Cambridge OCaml course. Will add in later if I feel like it.
\section{Testing}
Not critical to the Cambridge OCaml course. Will add in later if I feel like it.
\section{Streams and Laziness}
It is possible to generate infinite lists, also known as streams. To not use up memory, the next item in the list is only generated when a function is evaluated.
\begin{lstlisting}[language=Caml]
type 'a stream =
    Cons of 'a * (unit -> 'a stream)
(By convention, we type stream. Not that it is not a default type.*)
    
let rec from n =
  Cons(n, fun () -> from (n + 1))
(*The anonymous function works, because functions are already values and do not get evaluated until a value is fed in. Hence, the following iterations of from (n + 1) are not evaluated.*)

let nats = from 0
let hd (Cons (h, _)) = h
let tl (Cons (_, tf)) = tf ()
(*To evaluate tf, we pass a unit argument, (). A function that takes a unit argument is called a thunk.*)

let rec take n s =
  if n = 0 then []
  else hd s :: take (n - 1) (tl s)

let rec drop n s =
  if n = 0 then s
  else drop (n - 1) (tl s)
  
 (*Make sure you can figure out what take and drop does.*)
\end{lstlisting}
We can apply operations like arithmetic and mapping to streams as though they were lists, by using the following methods:
\begin{lstlisting}[language=Caml]
let rec square (Cons (h, tf)) = 
  Cons (h*h, fun () -> square (tf ()))

let rec sum (Cons (h1, tf1)) (Cons (h2, tf2)) =
  Cons (h1+h2, fun () -> sum (tf1 ()) (tf2 ()))
  
let rec map f (Cons (h, tf)) = 
	Cons (f h, fun () -> map f (tf ()))

let rec nats = Cons (0, fun () -> map (fun x -> x + 1) nats)
(*This was difficult to understand; to figure this out imagine nats = [0; 1; 2; 3;...]. Then map (fun x -> x + 1) nats = [1; 2; 3; 4;...].*)
\end{lstlisting}
One issue with this is that \verb|fun () -> ...| is not memoized, so every time it is called the value will be recalculated. OCaml has a high-level implementation to solve this problem called \verb|Lazy|.
\begin{lstlisting}[language=Caml]
let x = lazy (5 + 4) ;;
(*val x : int lazy_t = <lazy>*)

Lazy.force x;;
(*int = 9*)

x;;
(*int lazy_t = lazy 9*)
\end{lstlisting}
On instantiating, \verb|x| is of type \verb|int lazy_t| and returns a value of \verb|<lazy>|. We may use \verb|Lazy.force| to evaluate \verb|x|. After that, \verb|x| returns a new value, \verb|lazy 9|. This means that when \verb|x| is called again, it does not do the calculation.
\\\\
The lecture recommended trying to implement \verb|lazylist| for fibonacci numbers. No solution was given so here is my solution:
\begin{lstlisting}[language=Caml]
type 'a lazylist =
    Cons of 'a * 'a lazylist Lazy.t;;
    (*Lazy.t means lazy type, which is lazy_t, so you could type lazy_t instead!*)

let hd (Cons (h, _)) = h;;

let tl (Cons (_, tf)) = Lazy.force tf;;

let rec take n s = 
  if n = 0 then []
  else hd s :: take (n - 1) (tl s);;

let rec drop n s = 
  if n = 0 then s
  else drop (n - 1) (tl s);;

(*both methods for sum should work*)
let rec sum list1 list2 = 
  Cons ((hd list1) + (hd list2), lazy (sum (tl list1) (tl list2)));; 

let rec sum (Cons (h1, t1)) (Cons (h2, t2)) = 
  Cons (h1 + h2, lazy (sum (Lazy.force t1) (Lazy.force t2)));; 

let rec fibs =
  Cons(1, lazy (
      Cons(1, lazy (
          sum (tl fibs) fibs))))
    
let fib30lazyfast = take 30 fibs
let fib29lazyfast = take 29 fibs ;; 
\end{lstlisting}
\section{Balanced Trees}
Various data structures should be able to use the following functions:
\begin{lstlisting}[language=Caml]
module type Set = sig
  type 'a t
  val empty : 'a t
  val insert : 'a -> 'a t -> 'a t
  val mem : 'a -> 'a t -> bool
  (*mem stands for membership*)
end

module ListSet : Set = struct
  ...
end
module BstSet : Set = struct
  ...
end
module RbSet: Set = struct
  ...
end
\end{lstlisting}
\textbf{Binary Search Tree}
\begin{lstlisting}[language=Caml]
module BstSet = struct
(* AF: [Leaf] represents the empty set.
* [Node (l, v, r)] represents
* the set $AF(l) \union {v} \union
* AF(r)$.
* RI: for every [Node (l, v, r)],
* all the values in [l] are strictly less than
* [v], and all the values in [r] are strictly
* greater than [v]. *)
  type 'a t =
    | Leaf
    | Node of 'a t * 'a * 'a t
  let rec mem x = function
    | Leaf -> false
    | Node (l, v, r) ->
        if x < v then mem x l
        else if x > v then mem x r
        else true
  let rec insert x = function
    | Leaf -> Node (Leaf, x, Leaf)
    | Node (l, v, r) ->
        if x < v then Node(insert x l, v, r )
        else if x > v then Node(l, v, insert x r)
        else Node(l, x, r )
end

let x = BstSet.Leaf;;
let y = BstSet.insert 4 x |> BstSet.insert 5;; 
(*val y : int BstSet.t =
  BstSet.Node (BstSet.Leaf, 4, BstSet.Node (BstSet.Leaf, 5, BstSet.Leaf))*)
\end{lstlisting}
We will go through the time complexities of worst, average and best cases briefly.
\\\\
\textbf{Worst Case}

If you insert an ascending list of numbers, the next number will always be on the right side of the previous number so you have to climb down very far - $\mathcal{O}(n)$ - to reach the leaf!
\\\\
\textbf{Average Case}

A random set of numbers would lead to a rather 'balanced' tree. The height is $\log n$ so the time complexity is also $\mathcal{O}(\log n)$.
\\\\
\textbf{Best Case}

If you leaf the inputs properly you will definitely be able to get a tree with height $\log n$ and time complexity of $\mathcal{O}(\log n)$.
\\\\
In order to ensure that the time complexity is $\mathcal{O}(\log n)$, we may balance the tree. Common data structures that implement these are:
\begin{itemize}
	\item 2-3 Trees: all paths have the same length
	\item AVL Trees: length of shortest and longest path from any node differ by at most one
	\item Red-Black Trees: length of shortest and longest path from any node differ at most by factor of two
\end{itemize}
\textbf{Red-Black Trees}
\begin{itemize}
	\item Binary search tree with each node coloured red or black (1 bit)
	\item Conventions:
	\begin{itemize}
		\item Root is always black
		\item Empty leaves are black
	\end{itemize}
	\item Invariants:
	\begin{itemize}
		\item No red node has a red child
		\item Every path from root to leaf has the same number of black nodes
	\end{itemize}
	\item Thus the length of the longest path is at most twice the length of the shortest path (interleave red nodes between black nodes)
\end{itemize}
\textbf{Implementation}
\begin{lstlisting}[language=Caml]
module RbSet = struct
  type color = Red | Blk
  type 'a t =
    | Leaf
    | Node of (color * 'a t * 'a * 'a t)
  let empty = Leaf
  let rec mem x = function
    | Leaf -> false
    | Node (_, l, v, r) ->
        if x < v then mem x l
        else if x > v then mem x r
        else true
  let rec insert' x = function
    | Leaf -> Node(Red, Leaf, x, Leaf)
    | Node (col, l, v, r) ->
        if x < v
        then balance (col, (insert' x l), v, r)
        else if x > v
        then balance (col, l, v, (insert' x r))
        else Node (col, l, x, r)
  let insert x s =
    match insert' x s with
    | Node (_, l, v, r) -> Node(Blk, l, v, r)
    | Leaf -> failwith "impossible"
  let balance = function
    | (Blk, Node (Red, Node (Red, a, x, b), y, c), z, d)
    | (Blk, Node (Red, a, x, Node (Red, b, y, c)), z, d)
    | (Blk, a, x, Node (Red, Node (Red, b, y, c), z, d))
    | (Blk, a, x, Node (Red, b, y, Node (Red, c, z, d)))
      -> Node (Red, Node (Blk, a, x, b), y, Node (Blk, c, z, d))
    | t -> Node t
(*pattern matching makes the reassignments so easy!*)
\end{lstlisting}
\textbf{2-3 Trees}

Two types of nodes:
	\begin{itemize}
	\item 1 parent 2 children: obeys BST
	\item 2 parents 3 children (p1 p2 c1 c2 c3):
		\begin{itemize}
		\item \verb|c1 < p1|
		\item \verb|p1 <= c2 <= p2|
		\item \verb|p2 < c3|
		\end{itemize}
	\end{itemize}
\section{Mutable Data Types}
OCaml variables are immutable, but its references, fields and arrays are mutable.
\begin{lstlisting}[language=Caml]
let x = ref 0;;
val x : int ref = {contents = 0}
(*The type of ref 0 is int ref*)

x;;
- : int ref = {contents = 0}
!x;;
- : int = 0
x:=2;;
- : unit = ()
x;;
- : int ref = {contents = 2}
!x;;
- : int = 2
\end{lstlisting}
The binding of \verb|x| to the pointer is immutable, but the contents of the memory may change. This is useful in implementing a counter.
\begin{lstlisting}[language=Caml]
let next_val =
  let counter = ref 0
  in fun () ->
    incr counter;
    !counter
\end{lstlisting}
References may have aliases, i.e. another variable that points to the same location in memory. So if the value stored in that location changes, both variables when dereferenced will point to the new value.
\begin{lstlisting}[language=Caml]
let r1 = ref 3110;;
let r2 = ref 3110;;
r1 == r2;; (*false*)
r1 = r2;; (*true*)
\end{lstlisting}
\verb|==| tests for physical equality, i.e. if the pointers point to the same memory location. \verb|=| tests for structural equality, i.e. if the value they point to is the same. Most of the time you only need to compare values with structural inequality. 
\\\\
\textbf{Mutable Fields}
\begin{lstlisting}[language=Caml]
# type point = {x:int; y:int; mutable c:string};;
type point = {x:int; y:int; mutable c:string; }
# let p = {x=0; y=0; c="red"};;
val p : point = {x=0; y=0; c="red"}
# p.c <- "white";;
- : unit = ()
# p;;
val p : point = {x=0; y=0; c="white"}
# p.x <- 3;;
\end{lstlisting}
With this, we realise that refs are just implementations of mutables.
\begin{lstlisting}[language=Caml]
type 'a ref = { mutable contents: 'a }
let ref x = { contents = x }
let ( ! ) r = r.contents
let ( := ) r newval = r.contents <- newval
\end{lstlisting}
\textbf{Arrays}

Quick note: arrays are not lists!
\begin{lstlisting}[language=Caml]
# let v = [|0.; 1.|];;
val v : float array = [|0.; 1.|]
# v.(0) <- 5.;;
- : unit = ()
# v;;
- : float array = [|5.; 1.|]

while e1 do e2 done
for x=e1 to e2 do e3 done
for x=e1 downto e2 do e3 done
(*loops are back!*)
\end{lstlisting}
\section{Interpreters}
Compilers compile the source into the target program, before the target program can take in inputs and produce outputs. Interpreters can take in both the source program and inputs while producing outputs.
\\\\
This leads to possibly better error messages and better debuggers.
\\\\
\textbf{Mixed Compilation and Interpretation}

Source program compiles into an intermediate program. Then a virtual machine can take in the intermediate program and input while producing outputs.
\\\\
\textbf{Architecture}
	\begin{itemize}
	\item Front end: translate source code into abstract syntax trees (AST)
	\item Back end: translate AST into machine code
	\end{itemize}
Front end involves:
	\begin{itemize}
	\item Lexical analysis with lexer
	\item Syntactic analysis with parser
	\item Semantic analysis
	\end{itemize}
Functional languages are well-suited to implement compilers and interpreters
	\begin{itemize}
	\item Recursive functions can be converted into ASTs
	\item Recursive functions can be used to interpret ASTs
	\end{itemize}
To interpret the following AST \verb|Add (Int 5) (Add (Int 6) (Int 7))| We can use the following code:
\begin{lstlisting}[language=Caml]
type expr =
  | Int of int
  | Add of expr * expr
           
let rec eval e =
  if is_value e
  then e
  else eval (step e)
      
let is_value = function
  | Int _ -> true
  | Add _ -> false
    
let rec step = function
  | Int n -> failwith "Does not step"
  | Add(Int n1, Int n2) -> Int (n1+n2)
  | Add(Int n1, e2) -> Add(Int n1, step e2)
  | Add(e1, e2) -> Add(step e1, e2)
\end{lstlisting}
Another example: \verb|let x = 5 in x + 1|, which is equivalent to 

\verb|Let "x" (Int 5) (Add (Var "x") (Int 1))|
\begin{lstlisting}[language=Caml]
type expr =
  | Int of int
  | Add of expr * expr
  | Var of string
  | Let of string * expr * expr
           
let rec eval e =
  if is_value e
  then e
  else eval (step e)
let is_value = function
  | Int _ -> true
  | Add _ | Var _ | Let _ -> false
    
  (* [subst e v x] is e{v/x}, that is,
* [e] with [v] substituted for [x]. *)
let rec subst e v x = match e with
  | Var y -> if x=y then v else e
  | Int n -> Int n
  | Add(el,er) ->
      Add(subst el v x, subst er v x)
  | Let(y,ebind,ebody) ->
      let ebind'
        = subst ebind v x in
      if x=y
      then Let(y, ebind', ebody)
      else Let(y, ebind', subst ebody v x)

let rec step = function
  | Int n -> failwith "Does not step"
  | Add(Int n1, Int n2) -> Int (n1 + n2)
  | Add(Int n1, e2) -> Add (Int n1, step e2)
  | Add(e1, e2) -> Add (step e1, e2)
  | Var _ -> failwith "Unbound variable"
  | Let(x, Int n, e2) -> subst e2 (Int n) x
  | Let(x, e1, e2) -> Let (x, step e1, e2)
  \end{lstlisting}
 After all this work we can finally interpret 2 syntax trees...
 \\\\
 Btw the following sections are beyond Cambridge's OCaml syllabus so I will not make notes for them yet (This is a 3rd/4th year Cornell course!)
 \section{The Substitution Model}
 \section{The Environment Model}
 \section{Futures}
 \section{Monads}
\end{document}